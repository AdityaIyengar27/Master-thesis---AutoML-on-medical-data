
\clearpage
\pdfbookmark{Abstract}{Abstract}
\section*{Abstract}

\begin{center}
\begin{minipage}[t]{1.0\textwidth}
	In recent years, the complexity of applications where Neural Networks (NNs) can be used have evolved. There has been a shift from designing deep neural networks (DNNs) manually to automating the process of finding such designs. The field related to this automated approach is called AutoML. AutoML allows you to provide the labeled training data as input and receive an optimized model as output. It uses the concept of Neural Architecture Search (NAS) to search for the optimal DNN. Based on the selection criteria pertinent to the dataset, mutation of one or more selected model takes place which leads to generation of new models. This concept can also be used to build hardware specific DNN models, with variety of selection criteria to choose the best model. This thesis explores the design of an AutoML framework for an anonymous time series electrocardiogram (ECG) data to detect atrial fibrillation (AF) while reducing the number of parameters and maintaining good accuracy as the optimization criteria. 
\end{minipage}
\end{center}

\vfill

% \section*{Zusammenfassung}

% \begin{center}
% \begin{minipage}[t]{0.7\textwidth}
% 	\blindtext[2]
% \end{minipage}
% \end{center}

\vfill

\thispagestyle{empty}